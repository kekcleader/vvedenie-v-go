% !TEX root = ../vvedenie_v_go.tex
\section{Общая структура программы}

Программа на Го состоит из модулей (packages)\footnote{В языке Го
имеет место некоторая терминологическая путаница. Модули называются словом
package, а совокупности модулей компоненты называются словом module.}.
Модуль может либо целиком находиться в одном файле, либо быть разбит
на несколько файлов. В последнем случае все эти файлы должны
непосредственно находиться в одном и том же каталоге.

Всякий Го-файл начинается со слова \code{package}, за которым следует имя
модуля. Ниже в любом порядке следуют {\em объявления} импортируемых
модулей (\code{import}), констант (\code{const}), типов (\code{type}), переменных (\code{var}) и
процедур (\code{func})\footnote{Как {\em функциональные процедуры} (функции), так
и {\em собственно процедуры} (не возвращающие значения) обозначаются в Го словом func.}.
Однострочные комментарии отбиваются двумя дробями
«\code{//}», многострочные~"--- заключаются в скобки «\code{/*}» и «\code{*/}».
Ничего другого в Го-файле быть не должно. Все {\em операторы} программы
должны находиться внутри процедур, то есть нельзя размещать операторы
непосредственно в модуле, как это имеет место, например, в языке Питон.

\gocap{Схема программы на Го}{lst:go-skeleton}
\begin{gocode}
package имямодуля

import (
  "имя другого модуля"
  "ещё имя другого модуля"
)

const (
  имяКонстанты = 2
)

type (
  имяТипа = int64
  другоеИмяТипа = [4]byte
)

var (
  имяПеременной float32
  имяА, имяБ bool
)

func имяПроцедуры() {
  операторы
}

func ещёИмяПроцедуры() {
  операторы
}
\end{gocode}

Если в каталоге непосредственно находятся несколько Го-файлов, то все они
должны принадлежать к одному и тому же модулю. Это значит, что все эти
файлы должны начинаться со слова \code{package}, за которым следует одно
и то же имя. Напротив, имя {\em файла} не играет совершенно никакой роли.

Есть и одно исключение. Набор отдельных небольших программ (скриптов)
разрешается хранить в одном каталоге, но тогда каждый из таких файлов
необходимо компилировать и выполнять отдельно от других. Как следствие,
эти модули не могут импортировать друг друга.

Структура каталогов программы на Го может быть разветвлённой. А чтобы
компилятор смог разобраться, какой каталог является корневым, в этом
корневом каталоге должен лежать файл \code{go.mod}. Это позволяет запускать
компиляцию программы, даже находясь в одном из подкаталогов. В случае
с единичными файлами~"--- скриптами~"--- файл \code{go.mod} не используется.

Исполнение программы начинается с процедуры \code{main} модуля \code{main}\footnote{До этого, правда, запускаются процедуры init всех импортированных модулей, где они есть~"--- это подобно секциям BEGIN модулей языка Оберон.}.

Экспортированные объекты (константы, типы, глобальные переменные и процедуры) можно использовать в других модулях. Объект считается экспортированным, если имя начинается с заглавной буквы. То есть процедура \code{hello} не будет доступна из других модулей, а процедура \code{Hello}~"--- будет. Как и в языке Оберон, обращение к (экспортированным) объектам, находящимся в других модулях, всегда происходит через имя модуля и точку\footnote{Правда, в Го можно импортировать модуль и таким образом, чтобы его объекты были доступны и напрямую~"--- без уточняющего имени модуля, но к этой функции языка разработчики прибегают редко.}, например, \code{fmt.Print}\,. Это накладывает отпечаток и на именование объектов~"--- их имена часто не имеют префиксов.
