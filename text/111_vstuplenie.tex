% !TEX root = ../vvedenie_v_go.tex
\section{Го и Оберон}

Язык Го по своей синтаксической {\em структуре} практически идентичен языку Оберон. К существенным отличиям относятся наличие в языке Го некоторых конструкций, таких как срезы (slices), мапы, сопрограммы (goroutines), каналы и оператор \code{select}, возможность разбивать модуль (package) на несколько файлов, а также двухпроходной характер компилятора, позволяющий вызывать процедуры, объявленные ниже по тексту. В 2022~г. в Го появились и средства обобщённого программирования~"--- дженерики.

К особенностям, оказывающим ощутимое влияние на стиль программирования, относится возможность объявления переменных в любом месте процедуры, в том числе без явного указания типа данных, а также своеобразная модель ООП в Го, реализованная через типы-интерфейсы.

ООП в Обероне реализовано через расширения записных типов (record type extensions) и свзанные с типом процедуры (type-bound procedures)~"--- методы\footnote{Методы есть только в Обероне-2 и Компонентном Паскале. В оригинальном Обероне 1990 года и Обероне-07 используются поля записей процедурного типа.}. Аналогичная конструкция языка Го не обеспечивает полноценного наследования и позднего связывания. Методы в Го могут быть связаны с любыми типами, а не только с записными.

Если тип реализует некоторый интерфейс, в тексте программы этого видно не будет~"--- связывание интерфейса и его реализации происходит в ходе компиляции путём сравнения методов, описанных в интерфейсе, и методов, фактически связанных с типом. Эта крайне необычная особенность языка Го также сильно влияет на стиль программирования.

Нельзя не упомянуть о необычной возможности вставлять оператор присваивания сразу после слова \code{if}.

Циклы \code{WHILE}, \code{REPEAT}, \code{FOR} и \code{LOOP}\footnote{Бесконечный цикл LOOP (с выходом из середины цикла с помощью оператора EXIT) отсутствует в Обероне-07.} в языке Го обозначаются одним и тем же словом \code{for}~"--- различие программист должен научиться определять по количеству точек с запятыми на одной строчке со словом \code{for} и по наличию операторов \code{break} внутри цикла. Цикл \code{foreach}\footnote{Цикл foreach есть, например, в языке ПХП и используется для перебора всех элементов массива.} выражен в Го конструкцией \code{for i := range m \{ ... \}}.

В Го есть отдельный тип \code{string}. Строки не нуль-терминированы, неизменяемы и всегда имеют кодировку UTF-8.
